% THIS IS SIGPROC-SP.TEX - VERSION 3.1
% WORKS WITH V3.2SP OF ACM_PROC_ARTICLE-SP.CLS
% APRIL 2009
%
% It is an example file showing how to use the 'acm_proc_article-sp.cls' V3.2SP
% LaTeX2e document class file for Conference Proceedings submissions.
% ----------------------------------------------------------------------------------------------------------------
% This .tex file (and associated .cls V3.2SP) *DOES NOT* produce:
%       1) The Permission Statement
%       2) The Conference (location) Info information
%       3) The Copyright Line with ACM data
%       4) Page numbering
% ---------------------------------------------------------------------------------------------------------------
% It is an example which *does* use the .bib file (from which the .bbl file
% is produced).
% REMEMBER HOWEVER: After having produced the .bbl file,
% and prior to final submission,
% you need to 'insert'  your .bbl file into your source .tex file so as to provide
% ONE 'self-contained' source file.
%
% Questions regarding SIGS should be sent to
% Adrienne Griscti ---> griscti@acm.org
%
% Questions/suggestions regarding the guidelines, .tex and .cls files, etc. to
% Gerald Murray ---> murray@hq.acm.org
%
% For tracking purposes - this is V3.1SP - APRIL 2009

\documentclass{acm_proc_article-sp}

\begin{document}

\title{Data Fusion using Source Trustworthiness}
%
% You need the command \numberofauthors to handle the 'placement
% and alignment' of the authors beneath the title.
%
% For aesthetic reasons, we recommend 'three authors at a time'
% i.e. three 'name/affiliation blocks' be placed beneath the title.
%
% NOTE: You are NOT restricted in how many 'rows' of
% "name/affiliations" may appear. We just ask that you restrict
% the number of 'columns' to three.
%
% Because of the available 'opening page real-estate'
% we ask you to refrain from putting more than six authors
% (two rows with three columns) beneath the article title.
% More than six makes the first-page appear very cluttered indeed.
%
% Use the \alignauthor commands to handle the names
% and affiliations for an 'aesthetic maximum' of six authors.
% Add names, affiliations, addresses for
% the seventh etc. author(s) as the argument for the
% \additionalauthors command.
% These 'additional authors' will be output/set for you
% without further effort on your part as the last section in
% the body of your article BEFORE References or any Appendices.

\numberofauthors{3} %  in this sample file, there are a *total*
% of EIGHT authors. SIX appear on the 'first-page' (for formatting
% reasons) and the remaining two appear in the \additionalauthors section.
%
\author{
% You can go ahead and credit any number of authors here,
% e.g. one 'row of three' or two rows (consisting of one row of three
% and a second row of one, two or three).
%
% The command \alignauthor (no curly braces needed) should
% precede each author name, affiliation/snail-mail address and
% e-mail address. Additionally, tag each line of
% affiliation/address with \affaddr, and tag the
% e-mail address with \email.
%
% 1st. author
\alignauthor
Manual Heinkel \\
       \affaddr{UMass Amherst}\\
       \affaddr{140 Governors Drive}\\
       \affaddr{Amherst, Massachusetts}\\
       \email{heinkel@cs.umass.edu}
% 2nd. author
\alignauthor
Nicholas Monath \\
       \affaddr{UMass Amherst}\\
       \affaddr{140 Governors Drive}\\
       \affaddr{Amherst, Massachusetts}\\
       \email{nmonath@cs.umass.edu}
% 3rd. author
\alignauthor 
Lakshmi Nair \\
       \affaddr{UMass Amherst}\\
       \affaddr{140 Governors Drive}\\
       \affaddr{Amherst, Massachusetts}\\
       \email{lvnair@cs.umass.edu}
}
\date{8 March 2015}
% Just remember to make sure that the TOTAL number of authors
% is the number that will appear on the first page PLUS the
% number that will appear in the \additionalauthors section.

\maketitle
\begin{abstract}
To be filled in in the later report. 
\end{abstract}

% A category with the (minimum) three required fields
%\category{H.4}{Information Systems Applications}{Miscellaneous}
%A category including the fourth, optional field follows...
%\category{D.2.8}{Software Engineering}{Metrics}[complexity measures, performance measures]

%\terms{Theory}

%\keywords{ACM proceedings, \LaTeX, text tagging} % NOT required for Proceedings

\section{Introduction}
To be filled in in the later report.


\section{Related Work}
The advent of data sources on the web in the last two decades has sparked the need for automated methods of combining sources into a single source. As these data sources often contain noisy and incorrect data techniques for discerning which among a candidate set of conflicting values is correct are crucial. These techniques are at the heart of the problem of data fusion. The process is also referred to as truth discovery (cite), data integration (cite), etc. While there has been some early work on how this pertains to information extraction (cite) and question answering (wu:corroborating), the signifigant body of literature pertains to the construction of databases, particularly relational databases and (more recently) knowledge bases \cite{dong:data}. The problem of data fusion in relational databases typically consists of multiple phases \cite{bleiholder:data} \cite{li:truth}. Given a collection of data sources, which all contain data that is to be stored in a particular relation, the first step of the data fusion process is often the mapping of the data sources schema to the schema of underlying relation \cite{bleiholder:data}. It may also be necessary to at the instance level (what does this mean exactly) correct the mapping (luna dong); it means they represent the same object in different ways. This is a required step before the processing of the data. There has been some work done on how to effectively perform schema mapping (cite), but the problem is often largely domain dependent (cite). The process of selecting the accurate value for an attribute given the sources' values on the otherhand is not typically domain dependent. It is this problem which our work focuses on.

A naive approach to this problem is to resolve the conflicts by performing a majority vote amongst the sources. Early approaches such as {\sc TruthFinder} (cite)  extend this technique by incorporating a notion of source trustworthiness. (Insert sentence about how it is calculated). The estimation of the trustworthiness of data sources (how much one should trust the value presented by a source)  is a common thread amongst data fusion algorithms as noted by \cite{li:truth} \cite{waguih:truth}. (Add in sentences about Cosine, 2 and 3 estimate). Other approaches use probabilistic models. (Insert sentences about LTM) (Insert sentences about Roth et al).


Another way in which the data sources can be noisy is by copying values from one another. One of the novel contributions of (cite luna dong 2009) was that copying amongst data sources was captured by the model. 

In the last few years, data fusion has branched into a closely related problem referred to by some \cite{dong:data}(our prof) as \emph{knowledge fusion}. In this task the data are knowledge triples such as a pair of entities and a relation rather than a collection of attributes of a relation. In this way, knowledge fusion is to knowledge bases as data fusion is to relational databases. The problem of knowledge fusion is closely related to the problems of knowledge base completion/creation (cite), relation extraction (cite), etc. In this work, we focus on data fusion however.

Another related reserach area is the collection of labels from crowdsourcing \cite{nguyen:minimizing}. Particularly, techniques for using crowdsourcing to create labeled data sets and crowsourcing for multiple choice question answering. In these approaches the number of sources/users can be much greater than for usual data fusion problems \cite{li:truth} \cite{nguyen:minimizing}.



Other related areas are in crowdsourcing. Particuarly in the collating of noisy labels to form a single ground truth on a dataset. and in the domains of crowsourcing for multiple choice question answer. In these approaches the number of sources (users) is much greater than in the problem for creating relational databases.  

\bibliographystyle{abbrv}
\bibliography{references}  

\balancecolumns
\end{document}
